%----------------------------------------------------------------------------------------
%	DOCUMENT DEFINITION
%----------------------------------------------------------------------------------------

% we use article class because we want to fully customize the page
\documentclass[10pt,A4]{article}


%----------------------------------------------------------------------------------------
%	ENCODING
%----------------------------------------------------------------------------------------

%we use utf8 since we want to build from any machine
\usepackage[utf8]{inputenc}

%----------------------------------------------------------------------------------------
%	LOGIC
%----------------------------------------------------------------------------------------

\usepackage{xifthen}
\usepackage{calc}
\usepackage[hidelinks]{hyperref}
\usepackage{enumitem}
\usepackage{setspace}

%----------------------------------------------------------------------------------------
%	FONT
%----------------------------------------------------------------------------------------

% some tex-live fonts - choose your own

%\usepackage[defaultsans]{droidsans}
%\usepackage[default]{comfortaa}
%\usepackage{cmbright}
\usepackage[default]{raleway}
%\usepackage{fetamont}
%\usepackage[default]{gillius}
%\usepackage[light,math]{iwona}
%\usepackage[thin]{roboto}

% set font default
\renewcommand*\familydefault{\sfdefault}
\usepackage[T1]{fontenc}

% more font size definitions
\usepackage{moresize}

% font icons package
\usepackage{fontawesome}

%----------------------------------------------------------------------------------------
%	PAGE LAYOUT  DEFINITIONS
%----------------------------------------------------------------------------------------

%define page styles using geometry
\usepackage[a4paper]{geometry}

% for example, change the margins to 2 inches all round
\geometry{top=2cm, bottom=2cm, left=2cm, right=2cm}

% use customized header
\usepackage{fancyhdr}
\pagestyle{fancy}

%less space between header and content
\setlength{\headheight}{-5pt}

% customize header entries
\lhead{}
\rhead{}
\chead{}

%indentation is zero
\setlength{\parindent}{0mm}

%----------------------------------------------------------------------------------------
%	GRAPHICS DEFINITIONS
%----------------------------------------------------------------------------------------

% for drawing graphics and charts
\usepackage{tikz}
\usetikzlibrary{shapes, backgrounds}

% use to vertically center content
% credits to: http://tex.stackexchange.com/questions/7219/how-to-vertically-center-two-images-next-to-each-other
\newcommand{\vcenteredinclude}[1]{\begingroup
	\setbox0=\hbox{\includegraphics{#1}}%
	\parbox{\wd0}{\box0}\endgroup}

% use to vertically center content
% credits to: http://tex.stackexchange.com/questions/7219/how-to-vertically-center-two-images-next-to-each-other
\newcommand*{\vcenteredhbox}[1]{\begingroup
	\setbox0=\hbox{#1}\parbox{\wd0}{\box0}\endgroup}

%----------------------------------------------------------------------------------------
%	ICON-SET EMBEDDING
%----------------------------------------------------------------------------------------

% at this point we simplify our icon-embedding by simply referring to a set of png images.
% if you find a good way of including svg without conflicting with other packages you can
% replace this part
\newcommand{\icon}[2]{\colorbox{black}{\makebox(#2, #2){\textcolor{white}{\large\csname fa#1\endcsname}}}}	%icon shortcut
\newcommand{\icontext}[3]{ 						%icon with text shortcut
	\vcenteredhbox{\icon{#1}{#2}}\hspace{0.2cm}\vcenteredhbox{\textcolor{black}{#3}}
}

%----------------------------------------------------------------------------------------
% 	HEADER
%----------------------------------------------------------------------------------------

% remove top header line
\renewcommand{\headrulewidth}{0pt}

%remove botttom header line
\renewcommand{\footrulewidth}{0pt}

%remove pagenum
\renewcommand{\thepage}{}

%remove section num
\renewcommand{\thesection}{}


%----------------------------------------------------------------------------------------
%
% 	TIKZ GRAPHICS
%
%----------------------------------------------------------------------------------------

\newcounter{a}
\newcounter{b}
\newcounter{c}
\newcounter{barcount}

%----------------------------------------------------------------------------------------
% 	BAR CHART
%----------------------------------------------------------------------------------------

% draw a bar chart
% param 1: width
% param 2: height
% param 3: border color
% param 4: label text color
% param 5: label bg color
% param 6: cat 1 color
\newenvironment{barchart}[8]{

	\newcommand{\barwidth}{0.35}
	\newcommand{\barsep}{0.2}

	% param 1: overall percent
	% param 2: label
	% param 3: cat 1 percent
	% param 4: cat 2 percent
	% param 5: cat 3 percent
	\newcommand{\baritem}[5]{

		\pgfmathparse{##3+##4+##5}
		\let\perc\pgfmathresult

		\pgfmathparse{#2}
		\let\barsize\pgfmathresult

		\pgfmathparse{\barsize*##3/100}
		\let\barone\pgfmathresult

		\pgfmathparse{\barsize*##4/100}
		\let\bartwo\pgfmathresult

		\pgfmathparse{\barsize*##5/100}
		\let\barthree\pgfmathresult

		\pgfmathparse{(\barwidth*\thebarcount)+(\barsep*\thebarcount)}
		\let\barx\pgfmathresult

		\filldraw[fill=#6, draw=none] (0,-\barx) rectangle (\barone,-\barx-\barwidth);
		\filldraw[fill=#7, draw=none] (\barone, -\barx) rectangle (\barone+\bartwo,-\barx-\barwidth);
		\filldraw[fill=#8, draw=none] (\barone+\bartwo,-\barx ) rectangle (\barone+\bartwo+\barthree,-\barx-\barwidth);

		\node [label=180:\colorbox{#5}{\textcolor{#4}{##2}}] at (0,-\barx-0.175) {};
		\addtocounter{barcount}{1}
	}
	\begin{tikzpicture}
	\setcounter{barcount}{0}

}
{\end{tikzpicture}}

%----------------------------------------------------------------------------------------
% 	BUBBLE CHART
%----------------------------------------------------------------------------------------
\newcommand{\bubble}[5]{
	\definecolor{tmpcol}{RGB}{50,50,#5}
	% slice
	\filldraw[fill=black,draw=none] (#1,0.5) circle (#3);

	% outer label
	\node[label=\textcolor{black}{#4}] at (#1,0.7) {};
}

\newcommand{\bubbles}[2]{
	%reset counters
	\setcounter{a}{0}
	\setcounter{c}{150}
	\begin{tikzpicture}[scale=3]
	\foreach \p/\t in {#1} {
		\addtocounter{a}{1}
		\bubble{\thea/2}{\theb}{\p/25}{\t}{1\p0}
	}
	\end{tikzpicture}
}


%----------------------------------------------------------------------------------------
%	custom sections
%----------------------------------------------------------------------------------------

% create a coloured box with arrow and title as cv section headline
% param 1: section title
%
\newcommand{\cvsection}[1] {
	\textcolor{white}{\MakeUppercase{\textbf{#1}}}
}

\newcommand{\cvsect}[1]{
	\colorbox{black}{{\cvsection{#1}}}\\\\%
}

%----------------------------------------------------------------------------------------
% CUSTOM LOREM IPSUM
%----------------------------------------------------------------------------------------
\newcommand{\lorem}{Lorem ipsum dolor sit amet, consectetur adipiscing elit. Donec a diam lectus.}

%----------------------------------------------------------------------------------------
% ENTRY LIST
%----------------------------------------------------------------------------------------
\usepackage{tabularx}

\setlength{\tabcolsep}{0pt}
\newenvironment{entrylist}{%
	\begin{tabular*}{\textwidth}[t]{@{\extracolsep{\fill}}ll}
	}{%
	\end{tabular*}
}

\newcommand{\entry}[4]{%
	\parbox[t]{3.5cm}{%
		#1%
	}%
	&\parbox[t]{14cm}{%
		\textbf{#2}%
		\hfill%
		{\footnotesize \textbf{\textcolor{black}{#3}}}\\%
		#4%
	}\\\\}

\newcommand{\slashsep}{
	\hspace{2mm}/\hspace{2mm}
}

%----------------------------------------------------------------------------------------
%	DOCUMENT CONTENT
%----------------------------------------------------------------------------------------
\begin{document}

	%----------------------------------------------------------------------------------------
	%	TITLE HEADLINE
	%----------------------------------------------------------------------------------------
	\begin{minipage}[t]{0.65\textwidth}\hrule height 0pt width 0pt%
		\colorbox{black}{{\HUGE\textcolor{white}{\textbf{\MakeUppercase{Chuan-Che Yen}}}}}%

		\vspace{1mm}\LARGE{Senior Software Engineer}
	\end{minipage}%
	\begin{minipage}[t]{0.3\textwidth}\hrule height 0pt width 0pt%
		\small%
		\icontext{MapMarker}{12}{Kaohsiung / Taipei, Taiwan}\\
		\icontext{Phone}{12}{+886 929 051 511}\\
		\icontext{At}{12}{\href{mailto:rbx1219@gmail.com}{rbx1219@gmail.com}}\\
	\end{minipage}%

	% manage space by reducing font size
	\small%
	\vspace{0.5cm}

	%----------------------------------------------------------------------------------------
	%	SKILLS AND TECHNOLOGIES
	%----------------------------------------------------------------------------------------
	\cvsect{Experience}\nobreak
	\begin{entrylist}
		\entry
		{08/2020 - present}
		{Senior Software Engineer}
		{Synology Inc.}
		{I am responsible for developing and maintaining the QuickConnect feature, which enables users to access their NAS devices from anywhere without complex network settings\\
				\begin{itemize}[leftmargin=*]
					\vspace{-0.5\baselineskip}
					\item \textbf{QuickConnect} \small is a service that enables remote access to NAS devices through the internet.\\
					\begin{itemize}\normalsize
						\vspace{-1\baselineskip}
						\item Developing features as required, identifying performance bottlenecks, security assessment and adjustment, as well as performing SRE duties.
						\item Improving the use of \textit{hole punching} technique to reduce cloud traffic usage by \textbf{3\%} and enhance user experience.
						\item Approximately \textbf{58k} unauthorized serial numbers, also known as \textit{hacksyno}, have been blocked from using QuickConnect to initiate connections.
					\end{itemize}
					\item \textbf{DDNS} \small enables users to access their NAS using a domain name instead of an IP address.\\
					\begin{itemize}\normalsize
						\vspace{-1\baselineskip}
						\item Using QuickConnect HTTP long polling to reduce IP lookups and decrease server usage by \textbf{29\%}.
						\item Integrate a certificate application process for Synology products that can be used in different scenarios, in conjunction with QuickConnect and Let's Encrypt.
					\end{itemize}
					\item \textbf{ZTNA} \small refers to a ZTNA solution that utilizes Synology QuickConnect to improve connectivity and usability.\\
					\begin{itemize}\normalsize
						\vspace{-1\baselineskip}
						\item Using QuickConnect as a secure channel to initiate Zero Trust Network Access, and utilizing hole punching techniques to increase the possibility of direct connections.
						\item Implement a web application using Vue as the interface to integrate PC client and QuickConnect.
					\end{itemize}
				\end{itemize}
					}
		\entry
		{02/2019 – 08/2020}
		{Principle Software Engineer}
		{Hairdodo Inc.}
		{As the primary technical developer at Hairdodo, I am responsible for both front-end and back-end development, generating reports based on requirements, as well as maintaining and developing the Android app. Specifically, my responsibilities include:\\
		\begin{itemize}[leftmargin=*]
			\vspace{-0.5\baselineskip}
			\item \textbf{Service and Web UI}
			\begin{itemize} \normalsize
				\item Design RESTful API for data manipulation with framework \textit{Sails.js}.
				\item Build web application with framework \textit{React}.
				\item Apply lazy-loading to save \textbf{93\%} per month from Google Maps API billing.
			\end{itemize}
			% \item \textbf{Android}
			% \begin{itemize} \normalsize
			% 	\item Feature development in \textit{JAVA} / Google Pay Integration / Firebase integration.
			% \end{itemize}
			\item \textbf{StylePay} \small is a third-party payment platform that enables credit card transactions and has successfully increased the service fees for designers by up to \textbf{35\%}.
			\begin{itemize} \normalsize
				\item Supporting pre-binding credit cards and integration with Apple Pay / Google Pay for users.
				\item Integrating with banks and online invoicing systems to automate the process of generating reports, allocating rewards, issuing invoices, and reconciling accounts.
			\end{itemize}
		\end{itemize}
		}
		\entry
		{2015 – 06/2018}
		{Software Engineer}
		{Synology Inc.}
		{
			\begin{itemize}[leftmargin=*]
				\vspace{-0.5\baselineskip}
				\item Renovate live streaming mechanism By integrating FFmpeg / WebSocket / HTML5 and MSE(Media Source Extension)
			\end{itemize}}
	\end{entrylist}
	\\\\\\\\\\
	\cvsect{Summary}%
	\begin{minipage}[t]{0.4\textwidth}%
		With 8+ years of experience in software development, specializing in C/C++, JavaScript, and Python, I am currently working as a Senior Software Engineer at Synology Inc. \end{minipage}%
	\hfill
	\begin{minipage}[t]{0.5\textwidth}\hrule height 0pt width 0pt%
		\vspace{-10pt}%
		\begin{barchart}{10}{5.5}{red}{white}{black}{black}{blue}{red}
			\baritem{80}{Python}{80}{0}{0}
			\baritem{80}{JavaScript}{70}{0}{0}
			\baritem{80}{C/C++}{60}{0}{0}
			\baritem{80}{Golang}{60}{0}{0}
			\baritem{80}{Vue}{60}{0}{0}
			\baritem{80}{React}{50}{0}{0}
			\baritem{80}{Node.js}{50}{0}{0}
		\end{barchart}
	\end{minipage}%

	\begin{center}
			\bubbles{5/Git, 4/AWS, 4/Ansible, 5/Linux, 4/Teleport}{}
	\end{center}

	\cvsect{Education}
	\begin{entrylist}
		\entry
		{2013 – 2017}
		{Master's Degree}
		{Another Awesome University}
		{\lorem\lorem\lorem}
		\entry
		{2014}
		{ERASMUS}
		{Awesome University in Heaven}
		{\lorem\lorem}
		\entry
		{2007 – 2013}
		{Bachelor's Degree}
		{Awesome University in Heaven}
		{\lorem\lorem}
	\end{entrylist}
	\\\\
	\begin{minipage}[t]{0.3\textwidth}\hrule height 0pt width 0pt%
		\cvsect{Languages}
		\textbf{Czech} - native\\
		\textbf{English} - C2\\
		\textbf{Polish} - B1
	\end{minipage}%
	\hspace{0cm}
	\begin{minipage}[t]{0.3\textwidth}\hrule height 0pt width 0pt%
		\cvsect{Hobbies}
		I love... \lorem
	\end{minipage}%
	\hspace{2cm}
	\begin{minipage}[t]{0.3\textwidth}\hrule height 0pt width 0pt%
		\cvsect{Non profit}
		I help... \lorem
	\end{minipage}%
	% LANGUAGES

\end{document}